%%%%%%%%%%%%%%%%%
% This is an sample CV template created using altacv.cls
% (v1.3, 10 May 2020) written by LianTze Lim (liantze@gmail.com). Now compiles with pdfLaTeX, XeLaTeX and LuaLaTeX.
%
%% It may be distributed and/or modified under the
%% conditions of the LaTeX Project Public License, either version 1.3
%% of this license or (at your option) any later version.
%% The latest version of this license is in
%%    http://www.latex-project.org/lppl.txt
%% and version 1.3 or later is part of all distributions of LaTeX
%% version 2003/12/01 or later.
%%%%%%%%%%%%%%%%

%% If you need to pass whatever options to xcolor
\PassOptionsToPackage{dvipsnames}{xcolor}

%% If you are using \orcid or academicons
%% icons, make sure you have the academicons
%% option here, and compile with XeLaTeX
%% or LuaLaTeX.
% \documentclass[10pt,a4paper,academicons]{altacv}

%% Use the "normalphoto" option if you want a normal photo instead of cropped to a circle
% \documentclass[10pt,a4paper,normalphoto]{altacv}

\documentclass[10pt,a4paper,ragged2e,withhyper]{altacv}
%% AltaCV uses the fontawesome5 and academicons fonts
%% and packages.
%% See http://texdoc.net/pkg/fontawesome5 and http://texdoc.net/pkg/academicons for full list of symbols. You MUST compile with XeLaTeX or LuaLaTeX if you want to use academicons.

% Change the page layout if you need to
\geometry{left=1.25cm,right=1.25cm,top=1.5cm,bottom=1.5cm,columnsep=1.2cm}

% The paracol package lets you typeset columns of text in parallel
\usepackage{paracol}

% Change the font if you want to, depending on whether
% you're using pdflatex or xelatex/lualatex
\ifxetexorluatex
  % If using xelatex or lualatex:
  \setmainfont{Roboto Slab}
  \setsansfont{Lato}
  \renewcommand{\familydefault}{\sfdefault}
\else
  % If using pdflatex:
  \usepackage[rm]{roboto}
  \usepackage[defaultsans]{lato}
  % \usepackage{sourcesanspro}
  \renewcommand{\familydefault}{\sfdefault}
\fi

% Change the colours if you want to
\definecolor{SlateGrey}{HTML}{2E2E2E}
\definecolor{LightGrey}{HTML}{666666}
\definecolor{DarkPastelRed}{HTML}{450808}
\definecolor{PastelRed}{HTML}{8F0D0D}
\definecolor{GoldenEarth}{HTML}{E7D192}
\colorlet{name}{black}
\colorlet{tagline}{PastelRed}
\colorlet{heading}{DarkPastelRed}
\colorlet{headingrule}{GoldenEarth}
\colorlet{subheading}{PastelRed}
\colorlet{accent}{PastelRed}
\colorlet{emphasis}{SlateGrey}
\colorlet{body}{LightGrey}

% Change some fonts, if necessary
\renewcommand{\namefont}{\Huge\rmfamily\bfseries}
\renewcommand{\personalinfofont}{\footnotesize}
\renewcommand{\cvsectionfont}{\LARGE\rmfamily\bfseries}
\renewcommand{\cvsubsectionfont}{\large\bfseries}


% Change the bullets for itemize and rating marker
% for \cvskill if you want to
\renewcommand{\itemmarker}{{\small\textbullet}}
\renewcommand{\ratingmarker}{\faCircle}

%% sample.bib contains your publications
\addbibresource{main.bib}

\begin{document}
\name{Diogo Vala Correia}
\tagline{Electronics and Telecommunications Engineering MSc Student \@ UA}
%% You can add multiple photos on the left or right
\photoR{2.8cm}{photo}

\personalinfo{%
  % Not all of these are required!
  \email{dv.correia@ua.pt}
  \phone{+351 915800676}
  \mailaddress{Aveiro, Rua S Martinho 61, 3810-185 Aveiro}
  \location{Portugal}
  \homepage{www.dvcorreia.com}
  %\twitter{@twitterhandle}
  \linkedin{diogovalacorreia}
  \github{dvcorreia}
  %% You MUST add the academicons option to \documentclass, then compile with LuaLaTeX or XeLaTeX, if you want to use \orcid or other academicons commands.
  % \orcid{0000-0000-0000-0000}
  %% You can add your own arbtrary detail with
  %% \printinfo{symbol}{detail}[optional hyperlink prefix]
  % \printinfo{\faPaw}{Hey ho!}[https://example.com/]
  %% Or you can declare your own field with
  %% \NewInfoFiled{fieldname}{symbol}[optional hyperlink prefix] and use it:
  % \NewInfoField{gitlab}{\faGitlab}[https://gitlab.com/]
  % \gitlab{your_id}
}

\makecvheader
%% Depending on your tastes, you may want to make fonts of itemize environments slightly smaller
% \AtBeginEnvironment{itemize}{\small}

%% Set the left/right column width ratio to 6:4.
\columnratio{0.6}

% Start a 2-column paracol. Both the left and right columns will automatically
% break across pages if things get too long.
\begin{paracol}{2}
\cvsection{Volunteer Experience}

\cvevent{Logistics}{\href{https://eestec.net/cities/aveiro/}{EESTEC LC Aveiro}}{Dec 2019 -- Present}{Aveiro, Portugal}
\begin{itemize}
  \item Organized a PCB design and manufacturing workshop: sourced sponsors, speaking professors and developed the workshop \href{https://github.com/dvcorreia/ac2-detpic-pcb-shield}{support material}
\end{itemize}

\divider

\cvevent{Member of the Electronics Department}{\href{http://engeniusteam.web.ua.pt}{Engenius - UA Formula Student}}{Oct 2018 -- Sep 2019}{Aveiro, Portugal}
\begin{itemize}
  \item Developer of the \textit{Datumbazo} platform, created to process, store and display in real time data generated by the cars
  \item Planned the electrical distribution system for the Phoenix Combustion Car: cabling, distribution, connectors and redundancy system for fail-safe for both power and CAN bus 
  \item Maintained the embedded systems pipeline tools. Maintained the Latex documents templates for documentation and reports
\end{itemize}

\divider

\cvevent{Member of the Electrical Powertrain Department}{\href{http://engeniusteam.web.ua.pt}{Engenius - UA Formula Student}}{Oct 2017 -- Oct 2018}{Aveiro, Portugal}
\begin{itemize}
  \item Worked on the electric powertrain for the future SPYRO Electric Car: hazard and safety study for the electric power distribution, power distribution and battery monitoring system and CAN bus interface for the BMS system 
  \item Provided consulting in electronics and programming for the Combustion Powertrain department
  \item Worked in the CAN bus electronics and software development with the Telecommunications and Electronics Department
  \item Developed the Latex Documents Templates for documentation and reports
\end{itemize}

\cvsection{Projects}

\cvevent{\href{https://github.com/dvcorreia/greenscale}{Greenscale}}{Politecnico di Torino}{Dec 2018 -- Nov 2019}{Torino, Italy}
An IoT platform that enables monitoring of sensor information, allowing automated actions to be taken to control and maintain greenhouses. 
\begin{itemize}
\item Web interface for management and data visualization
\item Example code and libraries for NodeMCUs allowing a fast and plug and play experience connecting sensors and actuators
\item Built with micro-services architecture that scales the resources to the needs and use cases
\end{itemize}

\medskip

\cvsection{Education}



\cvevent{M.Sc.\ Electronics and Telecommunications Engineering}{Universidade de Aveiro}{Sept 2001 -- June 2002}{Aveiro, Portugal}

Dissertation title "Implementation of UHF RFID Retail inventory management system using EPC Global Architecture Framework":
\begin{itemize}
  \item RF survey of industrial shelves for EPC Class1 Gen2 passive tag readings
  \item Optimized RF operations, report contents and tag filtering through the LLRP protocol
  \item Architecture follows the supply-chain specifications in the EPC Global Framework
  \item Used Docker to containerize and orchestrate Fosstrak Java services
  \item Developed HTTP services (REST, websockets) using Golang
  \item Developed a web interface for inventory management with React JS
\end{itemize} 

\divider

\cvevent{Exange Student}{Politecnico di Torino}{Sept 2018 -- Jul 2019}{Torino Italy}

Studying at PoliTo made possible to extend the specializations offered in Aveiro with a set of subjects design for the automotive and telematic industries. Supported by the resident vehicle manufacturer group, Fiat, and university neighbor, General Motors, the education was nothing but the best I could get.
Subjects frequented at PoliTo:
\begin{itemize}
  \item Automotive infosystems (30/30)
  \item Statistical learning and neural networks (-/30)
  \item Programming for IoT applications (30L/30)
  \item Project Management (21/30)
  \item Signal processing: methods and algorithms (20/30)
\end{itemize}

\divider

\cvevent{B.Sc.\ in Electronics and Telecommunications Engineering}{Universidade de Aveiro}{Sep 2014 -- Jul 2017}{Aveiro, Portugal}

%% Switch to the right column. This will now automatically move to the second
%% page if the content is too long.
\switchcolumn

\cvsection{My Life Philosophy}

\begin{quote}
``Something smart or heartfelt, preferably in one sentence.''
\end{quote}

\cvsection{Most Proud of}

\cvachievement{\faTrophy}{Fantastic Achievement}{and some details about it}

\divider

\cvachievement{\faHeartbeat}{Another achievement}{more details about it of course}

\divider

\cvachievement{\faHeartbeat}{Another achievement}{more details about it of course}

\cvsection{Strengths}

\cvtag{Hard-working}
\cvtag{Eye for detail}\\
\cvtag{Motivator \& Leader}

\divider\smallskip

\cvtag{C++}
\cvtag{Embedded Systems}\\
\cvtag{Statistical Analysis}

\cvsection{Languages}

\cvskill{English}{5}
\divider

\cvskill{Spanish}{4}
\divider

\cvskill{German}{3}

%% Yeah I didn't spend too much time making all the
%% spacing consistent... sorry. Use \smallskip, \medskip,
%% \bigskip, \vpsace etc to make ajustments.
\medskip


% \divider

\cvsection{Referees}

% \cvref{name}{email}{mailing address}
\cvref{Prof.\ Alpha Beta}{Institute}{a.beta@university.edu}
{Address Line 1\\Address line 2}

\divider

\cvref{Prof.\ Gamma Delta}{Institute}{g.delta@university.edu}
{Address Line 1\\Address line 2}


\end{paracol}


\end{document}
