\documentclass[11pt,a4paper,roman]{moderncv}      
\usepackage[english]{babel}

\moderncvstyle{classic}                            
\moderncvcolor{green}                            

% character encoding
\usepackage[utf8]{inputenc}                     

% adjust the page margins
\usepackage[scale=0.75]{geometry}

% personal data
\name{Diogo}{Correia}


\phone[mobile]{+351915800676}               
\email{dv.correia@ua.pt}                             


\begin{document}

\recipient{To}{European Space Agency (ESA)}
\date{\today}
\opening{Dear Recruiter,}
\closing{Best regards,}
\makelettertitle


% The first paragraph of your job application letter should include information on why you are writing. Mention the job you are applying for and where you found the position. If you have a contact at the company, mention the person's name and your connection here.

I am writing this letter following my application to the \emph{Young Graduate Trainee for Microelectronics: Development of Integrated Circuits (12104)} opportunity.
The excitement I felt after finding this opportunity, was truly amazing. 

% The next section of your cover letter should describe what you have to offer the company. Make strong connections between your abilities and the requirements listed in the job posting. Mention specifically how your skills and experience match the job. Expand on the information in your resume, don't just repeat it.
% Try to support each statement you make with a piece of evidence. Use several shorter paragraphs or bullets rather than one large block of text, which can be difficult to read and absorb quickly.

The curiosity about space and the fundaments of the universe is a sweet and exciting topic for me. I still remember fondly the space encyclopedia I received has a kid from my mother, and despite being outdated, I still hold on and remember where we were 20 years ago.
My master's in electronics and telecommunications engineering may not be daily coupled with the front line of the universe scientific discovery, but that doesn't stop me from accompanying SpaceX's launches, the news from CERN and  exhaustively listening to the World Science Festival talks with Brian Greene (and even drag my veterinary housemate in the process).
The long lasting fight for mapping the unknown is truly romantic to my heart, much like David Deutsch said \textit{``the universe is not only queerer than we suppose, but queerer than we can suppose''}.

Throughout my university experience, I've learned digital systems, RF design and semiconductor device fabrication, to context a few, and tried to broaden my knowledge to different areas, like the automotive industry, neurology, supply chain, web, and cloud infrastructure. One thing in common to all, was the surprising necessity and fit for FPGA/ASIC based improvements.
This triggered the enthusiasm in FPGAs two months ago, making me grab my old audio synthesizer project and giving it go.

% Conclude your application letter by thanking the employer for considering you for the position. Include information on how you will follow-up. State that you will do so and indicate when (one week's time is typical). You may want to reduce the time between sending out your resume and follow up if you fax or email it.

I am fresh out of university and I recognize my lack of experience. I also acknowledge my passion for science, long last excitement for the universe and the unknown, and appreciation for the scientific method and community.

Thank you for your attention and time. I look forward to hearing from you. 

\vspace{0.5cm}

\makeletterclosing

\end{document}
